\chapter{Apprendimento Bayesiano}
Ci sono stati 2 modi per rappresentare la probabilità:
\begin{itemize}
    \item \textbf{soggettivo} (grado di conoscenza o probabilità Bayesiana): la probabilità
          di un evento sapendo che è accaduto un altro evento.
    \item \textbf{oggettivo} (Long-term Frequency): la probabilità di un evento è il limite
          della sua frequenza relativa in molti tentativi.
\end{itemize}
La probabilità in termini di limite sulle frequenze a lungo termine dei casi,
questo è un'interpretazione oggettiva perché si sfruttano delle formule e poi
abbiamo dei casi immutabili. Mi baso su parametri fissati e quindi calcolo la
probabilità in base ai parametri.

L'\textbf{apprendimento bayesiano} è un approccio all'apprendimento automatico che si basa
sulla probabilità bayesiana. L'idea è quella di utilizzare la probabilità per
rappresentare la conoscenza incerta e quindi aggiornare la probabilità in base
ai nuovi dati. Quindi non cerchiamo l'ipotesi che combacia con i dati ma cerchiamo
l'ipotesi più probabile.

L'apprendimento bayesiano è importante per due motivi:
\begin{enumerate}
    \item Si ha una manipolazione esplicita della probabilità rispetto ad altri
          approcci pratici di alcuni tipi di problemi di apprendimento.
    \item Fornisce una prospettiva utile per comprendere metodi di apprendimento
          che non manipolano esplicitamente la probabilità.
\end{enumerate}

Le regole del calcolo mi permettono di associare un numero a una probabilità ma
non un significato. La probabilità bayesiana è una probabilità che ha un significato.

Dal punto di vista delle funzionalità si ha che ogni esempio di training osservato
può aumentare o diminuire la probabilità di un'ipotesi. Inoltre, la conoscenza
pregressa può essere combinata con i dati di training per ottenere la probabilità
finale delle varie ipotesi.

Si hanno alcune difficoltà nell'apprendimento bayesiano:
\begin{itemize}
    \item La \textbf{probabilità bayesiana} è una probabilità \textbf{soggettiva}, quindi dipende
          dalla conoscenza pregressa.
    \item Si hanno costi \textbf{computazionali elevati}, perché si ha una complessità
          esponenziale rispetto al numero di variabili.
\end{itemize}

La statistica bayesiana è una probabilità soggettiva. Si ha una natura soggettiva
perché si ha una conoscenza incerta a cui viene associata una probabilità. Si
ha quindi una variabile casuale a cui associo una distribuzione di probabilità.
Lo stato delle cose è rappresentato da una variabile soggetta a probabilità.
Il parametro non è più certo ma è casuale.

Nell'apprendimento bayesiano la \textbf{miglior ipotesi} è quella che \textbf{massimizza} la probabilità.

Il punto centrale dell'apprendimento bayesiano è la \textbf{regola di Bayes},
la quale fornisce un metodo diretto per calcolare la probabilità di un'ipotesi
$h$ dato un insieme di dati $D$. La regola di Bayes è la seguente:
\begin{equation}
    P(h|D) = \frac{P(D|h) \cdot P(h)}{P(D)}
\end{equation}
dove:
\begin{itemize}
    \item $P(h|D)$ è la probabilità dell'ipotesi $h$ dato il dataset $D$
          (\textbf{posterior}), è la probabilità che l'ipotesi  dopo aver osservato
          l'evidenza.
    \item $P(D|h)$ è la probabilità del dataset $D$ dato l'ipotesi $h$
          (\textbf{likelihood}), è la probabilità di osservare l'evidenza $D$ dato che
          l'ipotesi $h$ è vera.
    \item $P(h)$ è la probabilità dell'ipotesi $h$ (\textbf{prior}), ovvero la
          probabilità dell'ipotesi prima di osservare l'evidenza. È un informazione
          a priori, il grado di fiducia rispetto all'ipotesi prima di osservare
          l'evidenza. Prima si fissa la prior e poi si osserva l'evidenza.
    \item $P(D)$ è la probabilità del dataset $D$ (\textbf{evidence}), ovvero
          la probabilità di osservare l'evidenza $D$. È la probabilità di come si
          comporta l'evidenza senza considerare l'ipotesi. L'evidenza è il dataset e 
          quindi al distribuzione dei dati.

          Quando ho due eventi indipendenti la probabilità congiunta è il prodotto
          delle due probabilità:
          \begin{equation}
              P(A, B) = P(A) \cdot P(B)
          \end{equation}
          Per calcolare la probabilità di evidenza $P(D)$ si utilizza la seguente
          formula:
          \begin{equation}
              P(D) = \sum_{h \in H} P(D, h) \cdot P(h)
          \end{equation}
\end{itemize}

Con la formula di Bayes si può calcolare la probabilità a posteriori. L'ipotesi
che meglio si adatta ai dati è quella che ha la probabilità a posteriori più alta.
Viene anche definita come \textbf{ipotesi MAP} (\textit{Maximum A Posteriori}).
\begin{equation}
    h_{MAP} = \text{argmax}_{h \in H} P(h|D) = \text{argmax}_{h \in H} \frac{P(D|h) \cdot P(h)}{P(D)} = \text{argmax}_{h \in H} P(D|h) \cdot P(h)
\end{equation}
dove $D$ rappresenta il dataset, argmax è l'ipotesi $h$ che massimizza la probabilità
a posteriori, si esclude dai conti $P(D)$ perché non stiamo cercando la probabilità
di $P(h|D)$, ma bensì stiamo cercardo quella che tra tutte ha probabilità massima, 
di conseguenzza $P(D)$ non modifica i risultati.

Spesso si assume che tutte le ipotesi siano equiprobabili (distribuzione uniforme),
quindi si ha la possibilità di semplificare i conti, perché, con lo stesso 
procedimento che abbiamo utilizzato sopra, possiamo escludere dal conto anche $P(h)$. 
In tal caso, la ricerca dell'\textbf{ipotesi MAP} prenderà il nome di ricerca
dell'\textbf{ipotesi ML} (\textit{Maximum Likelihood}), perché prenderemo $h$:
\begin{equation}
    h_{ML} = h_{MAP} =\text{argmax}_{h \in H} P(h|D) =\text{argmax}_{h \in H} P(D|h)
\end{equation}
potendo quindi trascurare $P(h)$ in quanto equivalente per tutte le ipotesi.

Si può utilizzare la formula di Bayes per specificare un algoritmo di apprendimento
molto semplice detto \textbf{algoritmo Brute-Force map learning}, che si articola
nei seguenti passi:
\begin{enumerate}
    \item Per ogni ipotesi $h \in H$ calcolo la probabilità a posteriori $P(h|D)$,
          utilizzato la formula di Bayes:
          \begin{equation}
              P(h|D) = \frac{P(D|h) \cdot P(h)}{P(D)}
          \end{equation}
    \item Restituisco l'ipotesi $h$ con probabilità a posteriori massima:
          \begin{equation}
              h_{MAP} = \text{argmax}_{h \in H} P(h|D)
          \end{equation}
\end{enumerate}
\begin{nota}
    La probabilità è un conteggio. Devo capire quante combinazioni di valori posso
    ottenere. Nello specifico si hanno: $\prod_{i=1}^n |D_i|$ combinazioni, dove
    $D_i$ è il dominio della variabile $X_i$.
\end{nota}
Il modo per semplificare l'apprendimento bayesiano è quello di usare il \textbf{naive bayes}.

Usando la formula di Bayes posso calcolare la probabilità dell'ipotesi $h$ dato
il training set $D$: $P(h|D)$. Tale probabilità prende il nome di \textbf{posterior},
posso calcolarla come segue:
\begin{equation}
    P(h|D) = \frac{P(D|h) \cdot P(h)}{P(D)}
\end{equation}
e non dalla tabella.

Viene indotta in maniera indiretta dalle probabilità presenti nella formula.

Supponiamo di conoscere $P(D| h)$ per conoscere \textit{cosa} posso calcolare:
\begin{equation}
    \sum_{i = 0}^{|H|} P(D, h_i) = \sum_{i = 0}^{|H|} P(D|h_i) \cdot P(h_i)
\end{equation}
\begin{equation}
    h_{MAP} = \text{argmax}_{h_i \in H} P(h_i|D) = \text{argmax}_{h_i \in H} P(D|h_i) \cdot P(h_i)
\end{equation}
Posso escludere $P(D)$ dal denominatore perché è una costante, se vogliamo
conoscere la posterior ci serve conoscere il denominatore.
\begin{definizione}[\textbf{Indipendeza condizionata}]
    \textbf{Indipendeza condizionata}:
    \begin{equation}
        P(X, Y|Z) = P(X|Z) \cdot P(Y|Z)
    \end{equation}
    la distribuzione è legata al fattore condizionante $Z$.
\end{definizione}
Questo mi permette di calcolare le singole probabilità sulle colonne del dataset
e poi moltiplicarle per ottenere quella totale:
\begin{equation}
    P(D| h) = \prod_{i = 0}^{|D|} P(d_i|h)
\end{equation}
dove $d_i$ è la colonna $i$-esima del dataset $D$.
\section{Naive Bayes}
Supponiamo di avere un dataset $D$ con $n$ attributi $a_1, \dots, a_n$ si cerca
di semplificare il calcolo utilizzando la proprietà dell'indipendenza condizionata:
\begin{equation}
    P(D|h) = \prod_{i = 0}^{|D|} P(d_i|h)
\end{equation}
Questo ci permette di definire il classificatore Bayesano naive è definito come:
\begin{equation}
    f_{NB} = \text{argmax}_{h_i \in H} P(h_i) \cdot \prod_{i = 0}^{|D|} P(d_i|h_i)
\end{equation}
Algoritmo naive bayes:
\begin{enumerate}
    \item Calcola le probabilità a priori $P(h_i)$.
    \item Calcola le probabilità condizionate $P(d_i|h_i)$.
    \item Calcola la probabilità a posteriori $P(h_i|D)$.
    \item Restituisci la classe con probabilità a posteriori maggiore.
\end{enumerate}
\begin{esempio}
    Dato il seguente dataset:
    \begin{table}[!ht]
        \centering
        \begin{tabular}{c|ccc|c}
            Esempi & A & B & C & Target \\ \hline
            $x_1$  & 1 & 1 & 1 & 0      \\ \hline
            $x_2$  & 0 & 1 & 1 & 0      \\ \hline
            $x_3$  & 1 & 0 & 1 & 1      \\ \hline
            $x_4$  & 0 & 0 & 0 & 1      \\ \hline
            $x_5$  & 0 & 1 & 0 & 1      \\ \hline
            $x_6$  & 1 & 1 & 0 & ?      \\
        \end{tabular}
    \end{table}
    $h_0 = (T = 0)$ allora abbiamo:
    $$P(T = 0 | A = 1, B = 1, C = 0) = \frac{P(A = 1, B = 1, C = 0 | T = 0) \cdot P(T = 0)}{P(x_6)}$$
    posso togliere il denominatore perché è una costante, e quindi uso solo quelle a numeratore.
    $$P(T = 0 | A = 1, B = 1, C = 0) = \\ P(A = 1 | T = 0) \cdot P(B = 1 | T = 0) \cdot P(C = 0 | T = 0) \cdot P(T = 0)$$
    A questo punto utilizzando i valori presenti nel dataset posso effettuare i calcoli:
    $$P(T = 0 | A = 1, B = 1, C = 0) = \frac{1}{2} \cdot 1 \cdot 0 \cdot \frac{2}{5} = 0$$
    Calcolo anche il caso $h_1 = (T = 1)$: $$P(T = 1 | A = 1, B = 1, C = 0) = \frac{P(A = 1, B = 1, C = 0 | T = 1) \cdot P(T = 1)}{P(x_6)}$$
    posso togliere il denominatore perché è una costante, e quindi uso solo
    quelle a numeratore. $$P(T = 1 | A = 1, B = 1, C = 0) = \\ P(A = 1 | T = 1) \cdot P(B = 1 | T = 1) \cdot P(C = 0 | T = 1) \cdot P(T = 1)$$
    ottenendo quindi: $$P(T = 1 | A = 1, B = 1, C = 0) = \frac{1}{3} \cdot \frac{1}{3} \cdot \frac{2}{3} \cdot \frac{3}{5} = \frac{2}{45}$$
    A questo punto calcoliamo $P(x_6)$ come: $$P(x_6) = \sum_{i = 0}^{|H|} P(x_6, h_i) = \sum_{i = 0}^{|H|} P(A =  1, B = 1, C = 0 | T = i) \cdot P(T = i)$$
    $$= P(A = 1 | T = 0) \cdot P(B = 1 | T = 0) \cdot P(C = 0 | T = 0) \cdot P(T = 0) + $$ $$gitP(A = 1 | T = 1) \cdot P(B = 1 | T = 1) \cdot P(C = 0 | T = 1) \cdot P(T = 1)$$
\end{esempio}
In generale per tutte le feature non vale l'\textbf{indipendenza condizionata}, ma
è stato dimostrato che la correttezza rimane invariata.

I principali problemi del \textbf{Naive Bayes} si hanno quando non si hanno abbastanza
dati per il training del modello. In questi casi risulta complesso stimare le
probabilità numericamente. Inoltre, se non osserviamo nessuna feature con una
particolare etichetta avremo una probabilità uguale a $0$. La soluzione per
quest'ultimo consiste nel aggiungere un piccolo numero di elementi in modo da
non rendere nulla la probabilità ma molto vicina allo zero.
\section{Gaussian Naive Bayes}
\textbf{Naive bayes} stima le probabilità basandosi sulle frequenze della classe di ciascuna
feature presente nel training set, quinfi funziona solo nel calcolo di distribuzioni
discrete. Se le feature sono continue, la probabilità di una feature può essere 
stimata usando la loro distribuzione continua oppure approssimando con una 
\textbf{distribuzione gaussiana}. 

Nella realtà si assumono che tutti gli attributi derivino da distribuzioni gaussiane 
per semplificare i conti.

Per classificare si calcolano sempre le probabilità, più precisamente la prior si
deriva dalla probabilità che una classe $c$ venga osservata tra le etichette del dataset. 
La likelihood si può calcolare usando la gaussiana:
\begin{equation}
    P(x_i|c_j) = \frac{1}{\sqrt{2 \pi \sigma_{i, j}^2}} \cdot e^{-\frac{(x_i - \mu_{i,j})^2}{2 \sigma_{i,j}^2}}
\end{equation}
dove $\mu_{i, j}$ rappresenta la media dell'i-esimo elemento della classe j-esima e $\sigma_{i, j}$
la deviazione standard dell'i-esimo elemento della classe j-esima, statistiche 
calcolate dal dataset. Dove $i$ sono le feature e $j$ sono le classi.

La previsione di un nuovo elemento può essere fatta come:
\begin{equation*}
    P(c_i | x) \stackrel{Bayes}{=} \frac{P(X | c_j) \cdot P(c_j)}{P(x)} \\ \stackrel{Naive}{=} \frac{\prod_i P(x_i | c_j) \cdot P(c_j)}{P(x)} \\ \stackrel{Gauss}{=} \frac{\prod_i \frac{1}{\sqrt{2\cdot \pi \cdot \sigma_{i, j}^2}} \cdot e^{- \frac{1}{2} \left(\frac{x_i - \mu_{i, j}}{\sigma_{i, j}}\right)^2} \cdot P(c_j)}{P(x)}
\end{equation*}
